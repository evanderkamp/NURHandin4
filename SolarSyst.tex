\section{Simulating the solar system}

For this question I could copy my leapfrog code from the working class without changes and the Euler I could easily change it to fit the problem by looking at the leapfrog code. 
Both the leapfrog algorithm and euler algorithm I made together with my sister Liz van der Kamp (s2135752) in the working class. 

\subsection*{a}

The code I wrote for this is:
\lstinputlisting[lastline=60]{NUR_handin4Q1.py}

First I got the positions and velocities as done in the example in the Handin description and I put them all in an array before converting the position values to AU like in the example. Then I plot the x-y and x-z positions at the "current" time (2021-12-07 10:00), see fig \ref{fig:fig1} and fig \ref{fig:fig2}

\begin{figure}[h!]
  \centering
  \includegraphics[width=0.9\linewidth]{NUR4Q1solsysxy.pdf}
  \caption{Plot corresponding to exercise 1a, showing  the x and y positions of the solar system on 2021-12-07 10:00 in AU.}
  \label{fig:fig1}
\end{figure} 


\begin{figure}[h!]
  \centering
  \includegraphics[width=0.9\linewidth]{NUR4Q1solsysxz.pdf}
  \caption{Plot corresponding to exercise 1a, showing  the x and z positions of the solar system on 2021-12-07 10:00 in AU.}
  \label{fig:fig2}
\end{figure} 


\subsection*{b and c}

The code I wrote for this is:
\lstinputlisting[firstline=61]{NUR_handin4Q1.py}

To get proper orbits when calculating the force between the sun and a planet, we first have the differential equation to calculate the acceleration of the planet: $d^2r/dt^2 = a = G*M*r/(|r|^2)^{3/2}$ where $G$ is the gravitational constant, $M$ is the mass of the sun, $r$ is a the position vector of the planet $(x,y,z)$ and $|r| = \sqrt{x^2 + y^2 + z^2}$. 
We also have $dr/dt = v$ where $v$ is the speed of the planet. To then properly integrate the orbit the best choice is to use a leapfrog method because that conserves energy by combining a forward method (which adds energy) and a backward method (which removes energy). 
It does so by first kicking the velocity of the planet half a step forward with a forward method and then calculating the following half steps using the previous half step. Then the positions get calcuted by using the previous position and the velocity from half a step forward (combining forward and backward).

With the leapfrog method I integrated the orbits over a time of 200 years with a time step of half a day first by shifting all the positions so the sun is at the origin and then converting all of the units to SI units so the acceleration calculation goes right. When saving the integrated positions I convert back to units of AU and plot the x-y positions and t versus z, see fig \ref{fig:fig3} and fig \ref{fig:fig4}

\begin{figure}[h!]
  \centering
  \includegraphics[width=0.9\linewidth]{NUR4Q1solsysxyorbits.pdf}
  \caption{Plot corresponding to exercise 1b, showing  the x and y positions of the solar system orbits integrated by the leapfrog algorithm in AU.}
  \label{fig:fig3}
\end{figure} 


\begin{figure}[h!]
  \centering
  \includegraphics[width=0.9\linewidth]{NUR4Q1solsystzorbits.pdf}
  \caption{Plot corresponding to exercise 1b, showing  the x and z positions of the solar system orbits integrated by the leapfrog algorithm in AU.}
  \label{fig:fig4}
\end{figure} 


